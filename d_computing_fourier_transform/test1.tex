%\subsection{The Fourier Transform for Periodic Signals}

\begin{frame}{The Fourier Transform for Periodic Signals: Introduction}
    In the previous section, we studied the Fourier transform representation, paying attention to aperiodic signals. We can also develop \ftrs~for periodic signals. This allows us to consider periodic and aperiodic signals in a unified context. We can construct the \ft of a periodic signal directly from its \fsr .
    
    Consider a signal $x(t)$  with \ft~ $X(j\omega)$ that is a single impulse of area $2\pi$ at $\omega=\omega_0$, i.e.,
    \begin{equation}
        X(j\omega) = 2\pi \delta(\omega-\omega_0)
    \end{equation}
\end{frame}
    
\begin{frame}    
    Let's determine the signal $x(t)$:
    \pause
    \mode<beamer>
    {
        \begin{equation*}
            \begin{split}
                x(t) &= \frac{1}{2\pi}\int_{-\infty}^{\infty} 2\pi \delta(\omega-\omega_0)e^{j\omega t} d\omega,\\
                &= e^{j\omega_0 t}.\\
            \end{split}
        \end{equation*}
        \pause
        More generally, if $X(j\omega)$ is of the form of a linear combination of impulses equally spaced in frequency, i.e.,
        \begin{equation}
             X(j\omega) = \sum_{k=-\infty}^{\infty}2\pi a_k \delta(\omega-k\omega_0)
        \end{equation}
        \pause
        then
        \begin{equation}
            x(t) = \sum_{k=-\infty}^{\infty}e^{jk\omega_0 t}.
        \end{equation}
        which is exactly the \fsr~ of a periodic signal. \\
        Thus, the \ft of a periodic signal with \fs~coefficients $\{a_k\}$ can be interpreted as a train of impulses occurring at the harmonically related frequencies and for which the area of the impulse at the $k$th harmonic frequency $k\omega_0$ is $2\pi$ times the $k$th \fs~coefficient $a_k$.
    }

\end{frame} 


\begin{frame}[plain]
    Example: Find the \ft of the signal whose \fs~coefficients are 
    \begin{equation*}
        a_k = \frac{\sin k \omega_0 T_1}{\pi k}.
    \end{equation*}
\end{frame}

\begin{frame}[plain]
    Example:  Find the \ft of 
    \begin{equation*}
        x(t) = \sin \omega_0 t.
    \end{equation*}
    and
    \begin{equation*}
        x(t) = \cos \omega_0 t.
    \end{equation*}    
\end{frame}

\begin{frame}[plain]
    Example: Find the \ft of the impulse train
    \begin{equation*}
        x(t) = \sum_{k=-\infty}^{\infty} \delta(t-kT).
    \end{equation*}
\end{frame}