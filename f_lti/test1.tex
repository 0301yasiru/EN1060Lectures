\section{Revisiting Fourier Series}
\begin{frame}{Convolution}
    \begin{enumerate}
      \item In representing and analyzing LTI systems, our approach has been to decompose the system inputs into a liner combination of basic signals and exploit the fact that for a linear system, the response is the same linear combination of the responses to the basic inputs.
      \item The convolution sum and the convolution integral req out of the particular choice of the basic signals, delayed unit impulses.
      \item This choice has the advantage that for systems that are time invariant in addition to being linear, once the response to an impulse at one time position is known, then the response id know at all time positions. 
    \end{enumerate}
\end{frame}

\begin{frame}{Complex Exponentials with Unity Magnitude as Basic Signals}
    \begin{enumerate}
      \item When we select complex exponential with unity magnitude as the basic signals, the decomposition of this form of a periodic signal is the Fourier series. 
      \item For aperiodic signals, it becomes the Fourier transform.
      \item In latter lectures, we will generalize thsi representation to Laplace tangram for continuous-time signals and $z$-tansfrom for discrete-time signals. 
    \end{enumerate}
\end{frame}

\begin{frame}[plain]
    Consider a linear system
            \column{0.68\textwidth}
                \tikzstyle{int}=[draw, fill=pink!60, minimum size=2em]
\tikzstyle{init} = [pin edge={to-,thin,black}]

\begin{tikzpicture}[node distance=2.5cm,auto,>=latex']
	
	\begin{scope}[yshift=-2cm,
		start chain,
		node distance=5mm,
		every node/.style={int, draw,on chain,join,inner sep=2mm},
		every join/.style={->}
		]
		\node[draw=none, fill=none]  {$u[n]$};	
		\node { $h[n]$};
		\node[draw=none, fill=none]   {$s[n]$};
	\end{scope}
	
	
\end{tikzpicture} 

            \column{0.3\textwidth}    
\end{frame}