% --------------------------------------------------------------
% This is all preamble stuff that you don't have to worry about.
% Head down to where it says "Start here"
% --------------------------------------------------------------

\documentclass[11pt]{article}
\usepackage[margin=1in]{geometry}
\usepackage{amsmath,amsthm,amssymb}
%\usepackage{multicol}
\usepackage{graphicx}
%\usepackage{fixltx2e}
%\usepackage{amsmath}

\usepackage{tikz}
\usepackage{pgfplots}
\usepackage{heuristica}
\usepackage[heuristica,vvarbb,bigdelims]{newtxmath}
\usepackage[T1]{fontenc}
\usepackage[inline]{enumitem}
\usepackage{subcaption}



%\everymath{\displaystyle}

\newcommand\ft{Fourier transform}
\newcommand\ift{inverse Fourier transform}
\newcommand\ftr{Fourier transform representation}
\newcommand\fs{Fourier series}
\newcommand\fsr{Fourier series representation}
\newcommand\xt{$x(t)$}
\newcommand\xo{$X(j\omega)$}
\newcommand\dtfs{discrete-time Fourier series}

\title{\Large Department of Electronic and Telecommunication Engineering\\University of Moratuwa\\Sri Lanka\\{\LARGE \bf \textsc{EN1060 Signals and Systems: Tutorial 06 Discreet-Time Fourier Transfrom \footnote{All the questions are from Oppenheim \emph{et al.} chapter 4.}}}}

\date{\vspace{-0.2in}\today}


\newcommand{\N}{\mathbb{N}}
\newcommand{\Z}{\mathbb{Z}}

\begin{document}



\maketitle
\noindent \tikz \draw (0,0) -- (\textwidth,0);

\begin{enumerate}
    % Q01 Oppenheim et al. p. 361
    \item Write the synthesis and analysis equation for the discrete-time Fourier transform.

    % Q01 Oppenheim et al. p. 362
    \item Consider the signal
        \begin{equation*}
            x[n] = a_nu[n], |a| <1.
        \end{equation*}
        \begin{enumerate}
            \item Express its DTFT $X\left(e^{j\omega}\right)$.
            \item Sketch the magnitude and phase of $X\left(e^{j\omega}\right)$ when $a>0$.
            \item Sketch the magnitude and phase of $X\left(e^{j\omega}\right)$ when $a<0$.
        \end{enumerate}
    \item Express the DTFT of
        \begin{equation*}
            x[n] = a^{|n|}, |a|<1,
        \end{equation*}
        and sketch.
    \item Express the DTFT of
        \begin{equation*}
            x[n] = \begin{cases}1, & |n| \leq N_1\\0, & |n| > N_1\end{cases}
        \end{equation*}
        and sketch it for $N_1 = 2$.
    \item Express and sketch the DTFT of $x[n] = \cos \omega_0 n$.
    \item Express and sketch the DTFT of the discrete-time periodic impulse train $x[n] = \sum_{k=-\infty}^{+\infty}\delta[n - kN]$.

    % Oppenheim q5.21
    \item Compute the \ft of each of the following signals:
    	\begin{enumerate}
    		\item $x[n] = u[n-2] - u[n-1]$
    		\item $x[n] = \left(\frac{1}{2}\right)^{-n}u[-n-1]$
    		\item $x[n] = \left(\frac{1}{3}\right)^{|n|}u[-n-2]$
    		\item $x[n] = \begin{cases}
    		n, & -3 \leq n \leq 3\\ 0, &\text{otherwise}.
    		\end{cases}$
    	\end{enumerate}
    	
    % Oppenheim q5.22
    \item The following are the Fourier transforms of discrete-time signals. Determine the signal corresponding to each transform.
	    \begin{enumerate}
	    	\item $X\left(e^{j\omega}\right) = \begin{cases}
	    	1, &\frac{\pi}{4} \leq |\omega| \leq \frac{3\pi}{4} \\ 0, & \frac{3\pi}{4} \leq |\omega| \leq \pi, 0 \leq |\omega| < \frac{\pi}{4} \end{cases}$
	    	\item $X\left(e^{j\omega}\right) = 1 + 3e^{-j\omega} + 2e^{-j2\omega} -4e^{-j3\omega} + e^{-j10\omega}$
	    	\item $X\left(e^{j\omega}\right) = e^{-j\omega/2}$ for $-\pi \leq \omega \leq \pi$
	    	\item $X\left(e^{j\omega}\right) = \frac{e^{-j\omega} - \frac{1}{5}}{1-\frac{1}{5}e^{-j\omega}}$	    	
	    \end{enumerate}

    % Oppenheim p. 373
    \item Show that 
        \begin{equation*}
            X\left(e^{j(\omega + 2\pi)}\right) = X\left(e^{j\omega}\right).
        \end{equation*}
        Is the continuous-time Fourier transform always periodic as the DTFT?
    % Oppenheim p. 379
    \item Fig. \ref{fi:example59} shows a sequence $x[n]$ adn $y[n]$. 
    \begin{enumerate}
    	\item Express $x[n]$ using subsampled versions of $y[n]$.
    	\item Express $Y\left(e^{j\omega}\right)$.
    	\item Hence, express $X\left(e^{j\omega}\right)$.    	
    \end{enumerate}
    



 	\begin{figure}
		\begin{tikzpicture}[scale=0.8, every node/.append style={font=\scriptsize}]

	\begin{filecontents}{xn.dat}
		n xn
		-1 0
		0 1
		1 2
		2 1
		3 2
		4 1
		5 2
		6 1
		7 2
		8 1
		9 2
		10 0
		11 0		
	\end{filecontents}
	
	\begin{filecontents}{yn.dat}
		n xn
		-1 0
		0 1
		1 1
		2 1
		3 1
		4 1	
		5 0
		6 0
		7 0
		8 0
		9 0
		10 0
		11 0
	\end{filecontents}	
	
	\begin{filecontents}{y2n.dat}
		n xn
		-1 0
		0 1
		1 0
		2 1
		3 0
		4 1	
		5 0
		6 1
		7 0
		8 1
		9 0
		10 0
		11 0
	\end{filecontents}	
	
	\begin{filecontents}{2y2nminus1.dat}
		n xn
		-1 0
		0 0
		1 2
		2 0
		3 2
		4 0	
		5 2
		6 0
		7 2
		8 0
		9 2
		10 0
		11 0
	\end{filecontents}		
		
	

    \begin{axis}[
    		name=axis1,
		y=0.5cm,
		x=0.5cm,
		 clip=false,
		 xmin=-2,xmax=12,
		 xlabel= $n$,
		 ylabel={$x[n]$},
		 ymin=0,ymax=3.5,
		 axis lines=middle,
         	xtick={ -1, 0, 1, 2, 3, 4, 5, 6, 7, 8, 9},
         	%xticklabels={$-\pi$, $\pi$, $2\pi$},
		%ytick={},
		 %yticklabels=\empty,
		 every axis x label/.style={at={(ticklabel* cs:1.05)}, anchor=west,},
		every axis y label/.style={at={(ticklabel* cs:1.05)}, anchor=south,},
     ]
		%\addplot [red, smooth, very thick, mark=none] table [x={omega}, y={X}] {./h_dtft/figures/dtft_square_pulse_N1_2.dat};
		\addplot [red, ycomb,  thick, mark=*] table [x={n}, y={xn}] {./xn.dat};
    \end{axis}

\pause
\mode<beamer>
{
%
    \begin{axis}[
    		name=axis2,
    		at={($(axis1.south)+(0cm,-1cm)$)},anchor= north,
		y=0.5cm,
		x=0.5cm,
		 clip=false,
		 xmin=-2,xmax=12,
		 xlabel= $n$,
		 ylabel={$y[n]$},
		 ymin=0,ymax=2.5,
		 axis lines=middle,
         	xtick={-15, -12, -9, -6, -3, 0, 3, 6, 9, 12, 15},
         	%xticklabels={$-\pi$, $\pi$, $2\pi$},
		%ytick={},
		 %yticklabels=\empty,
		 every axis x label/.style={at={(ticklabel* cs:1.05)}, anchor=west,},
		every axis y label/.style={at={(ticklabel* cs:1.05)}, anchor=south,},
     ]
		%\addplot [red, smooth, very thick, mark=none] table [x={omega}, y={X}] {./h_dtft/figures/dtft_square_pulse_N1_2.dat};
		\addplot [red, ycomb,  thick, mark=*] table [x={n}, y={xn}] {./yn.dat};
    \end{axis}
%
}


\pause
\mode<beamer>
{
%
    \begin{axis}[
    		name=axis3,
    		at={($(axis2.south)+(0cm,-1cm)$)},anchor= north,
		y=0.5cm,
		x=0.5cm,
		 clip=false,
		 xmin=-2,xmax=12,
		 xlabel= $n$,
		 ylabel={$y_{(2)}[n]$},
		 ymin=0,ymax=2.5,
		 axis lines=middle,
         	xtick={-15, -12, -9, -6, -3, 0, 3, 6, 9, 12, 15},
         	%xticklabels={$-\pi$, $\pi$, $2\pi$},
		%ytick={},
		 %yticklabels=\empty,
		 every axis x label/.style={at={(ticklabel* cs:1.05)}, anchor=west,},
		every axis y label/.style={at={(ticklabel* cs:1.05)}, anchor=south,},
     ]
		%\addplot [red, smooth, very thick, mark=none] table [x={omega}, y={X}] {./h_dtft/figures/dtft_square_pulse_N1_2.dat};
		\addplot [red, ycomb,  thick, mark=*] table [x={n}, y={xn}] {./y2n.dat};
    \end{axis}
%
}


\pause
\mode<beamer>
{
%
    \begin{axis}[
    		name=axis4,
    		at={($(axis3.south)+(0cm,-1cm)$)},anchor= north,
		y=0.5cm,
		x=0.5cm,
		 clip=false,
		 xmin=-2,xmax=12,
		 xlabel= $n$,
		 ylabel={$2y_{(2)}[n-1]$},
		 ymin=0,ymax=3.5,
		 axis lines=middle,
         	xtick={-15, -12, -9, -6, -3, 0, 3, 6, 9, 12, 15},
         	%xticklabels={$-\pi$, $\pi$, $2\pi$},
		%ytick={},
		 %yticklabels=\empty,
		 every axis x label/.style={at={(ticklabel* cs:1.05)}, anchor=west,},
		every axis y label/.style={at={(ticklabel* cs:1.05)}, anchor=south,},
     ]
		%\addplot [red, smooth, very thick, mark=none] table [x={omega}, y={X}] {./h_dtft/figures/dtft_square_pulse_N1_2.dat};
		\addplot [red, ycomb,  thick, mark=*] table [x={n}, y={xn}] {./2y2nminus1.dat};
    \end{axis}
%
}
\end{tikzpicture} 
		\caption{Sequences}
		\label{fi:example59}
	\end{figure}

	\item Let $X\left(e^{j\omega}\right)$ denote the \ft of the signal shown in Fig. \ref{fi:p523}. Perform the following calculations wihtout explicitly evaluating $X\left(e^{j\omega}\right)$:
 	\begin{figure}
		\begin{tikzpicture}[xscale=0.5]




	\def\nmin{-12}
	\def\nmax{14}	
	
	\begin{scope}	
		\def\x{{0, 0, 0, 0, 0, 0, 0, 0, 0, -1, 0, 1, 2, 1, 0, 1, 2, 1, 0, -1, 0, 0, 0, 0, 0, 0, 0}}	
		\draw (\nmin-1, 0) -- (\nmax+1, 0) node[anchor=west] {\scriptsize $n$};
		\draw (0,0) -- ++(0, 2.5) node [anchor=south] {\scriptsize $x[n]$};
% 		\foreach \n/\l in {-1/{-N_1}, 1/0, 3/{N_1}}
% 		{
% 			\node at (\n/4, 0) [anchor=north] {\scriptsize $\l$};
% 		}
		\foreach \n in {-3, -2, ..., 8}
		{
			\node at (\n, 0) [anchor=north] {\scriptsize $\n$};
		}
		\node at (-0.75,1) [anchor=west] {\scriptsize $1$};
		\node at (-0.75,2) [anchor=west] {\scriptsize $2$};
		
		\foreach \n in {0,1, ..., 26}
		{
			\pgfmathsetmacro\xn{\x[\n]}
			\pgfmathparse{\xn == 0 ? 1 : 0}	
			\ifthenelse{\pgfmathresult>0}
			{
				\draw[brown, fill=brown] (\n+ \nmin,  0) circle (1pt);

			}
			{
				\draw[brown, thick, fill=brown]  (\n + \nmin, 0) -- ++(0, \xn) circle (1pt);% node[anchor=east] {\scriptsize $\xn$};			
				
			}
		}
	\end{scope}		
	
	
\end{tikzpicture}
		\caption{Sequences}
		\label{fi:p523}
	\end{figure}
	\begin{enumerate}
		\item Evaluate $X\left(e^{j0}\right)$
		\item Find $\angle X\left(e^{j\omega}\right)$
		\item Evaluate $\int_{-\pi}^{\pi}X\left(e^{j\omega}\right)d\omega$
		\item Find $X\left(e^{j\pi}\right)$
		\item Determine and sketch the signal whose \ft is $\mathfrak{Re}\{x(\omega)\}$
		\item Evaluate $\int_{-\pi}^{\pi}\left|X\left(e^{j\omega}\right)\right|^2d\omega$
		\item Evaluate $\int_{-\pi}^{\pi}\left|\frac{dX\left(e^{j\omega}\right)}{d\omega}\right|^2d\omega$		
	\end{enumerate}
	

    
\end{enumerate}

\end{document} 